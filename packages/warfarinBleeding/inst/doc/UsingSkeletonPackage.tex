\documentclass[]{article}
\usepackage{lmodern}
\usepackage{amssymb,amsmath}
\usepackage{ifxetex,ifluatex}
\usepackage{fixltx2e} % provides \textsubscript
\ifnum 0\ifxetex 1\fi\ifluatex 1\fi=0 % if pdftex
  \usepackage[T1]{fontenc}
  \usepackage[utf8]{inputenc}
\else % if luatex or xelatex
  \ifxetex
    \usepackage{mathspec}
  \else
    \usepackage{fontspec}
  \fi
  \defaultfontfeatures{Ligatures=TeX,Scale=MatchLowercase}
\fi
% use upquote if available, for straight quotes in verbatim environments
\IfFileExists{upquote.sty}{\usepackage{upquote}}{}
% use microtype if available
\IfFileExists{microtype.sty}{%
\usepackage{microtype}
\UseMicrotypeSet[protrusion]{basicmath} % disable protrusion for tt fonts
}{}
\usepackage[margin=1in]{geometry}
\usepackage{hyperref}
\hypersetup{unicode=true,
            pdftitle={Using the package skeleton for patient-level prediction studies},
            pdfauthor={Jenna M. Reps},
            pdfborder={0 0 0},
            breaklinks=true}
\urlstyle{same}  % don't use monospace font for urls
\usepackage{color}
\usepackage{fancyvrb}
\newcommand{\VerbBar}{|}
\newcommand{\VERB}{\Verb[commandchars=\\\{\}]}
\DefineVerbatimEnvironment{Highlighting}{Verbatim}{commandchars=\\\{\}}
% Add ',fontsize=\small' for more characters per line
\usepackage{framed}
\definecolor{shadecolor}{RGB}{248,248,248}
\newenvironment{Shaded}{\begin{snugshade}}{\end{snugshade}}
\newcommand{\KeywordTok}[1]{\textcolor[rgb]{0.13,0.29,0.53}{\textbf{#1}}}
\newcommand{\DataTypeTok}[1]{\textcolor[rgb]{0.13,0.29,0.53}{#1}}
\newcommand{\DecValTok}[1]{\textcolor[rgb]{0.00,0.00,0.81}{#1}}
\newcommand{\BaseNTok}[1]{\textcolor[rgb]{0.00,0.00,0.81}{#1}}
\newcommand{\FloatTok}[1]{\textcolor[rgb]{0.00,0.00,0.81}{#1}}
\newcommand{\ConstantTok}[1]{\textcolor[rgb]{0.00,0.00,0.00}{#1}}
\newcommand{\CharTok}[1]{\textcolor[rgb]{0.31,0.60,0.02}{#1}}
\newcommand{\SpecialCharTok}[1]{\textcolor[rgb]{0.00,0.00,0.00}{#1}}
\newcommand{\StringTok}[1]{\textcolor[rgb]{0.31,0.60,0.02}{#1}}
\newcommand{\VerbatimStringTok}[1]{\textcolor[rgb]{0.31,0.60,0.02}{#1}}
\newcommand{\SpecialStringTok}[1]{\textcolor[rgb]{0.31,0.60,0.02}{#1}}
\newcommand{\ImportTok}[1]{#1}
\newcommand{\CommentTok}[1]{\textcolor[rgb]{0.56,0.35,0.01}{\textit{#1}}}
\newcommand{\DocumentationTok}[1]{\textcolor[rgb]{0.56,0.35,0.01}{\textbf{\textit{#1}}}}
\newcommand{\AnnotationTok}[1]{\textcolor[rgb]{0.56,0.35,0.01}{\textbf{\textit{#1}}}}
\newcommand{\CommentVarTok}[1]{\textcolor[rgb]{0.56,0.35,0.01}{\textbf{\textit{#1}}}}
\newcommand{\OtherTok}[1]{\textcolor[rgb]{0.56,0.35,0.01}{#1}}
\newcommand{\FunctionTok}[1]{\textcolor[rgb]{0.00,0.00,0.00}{#1}}
\newcommand{\VariableTok}[1]{\textcolor[rgb]{0.00,0.00,0.00}{#1}}
\newcommand{\ControlFlowTok}[1]{\textcolor[rgb]{0.13,0.29,0.53}{\textbf{#1}}}
\newcommand{\OperatorTok}[1]{\textcolor[rgb]{0.81,0.36,0.00}{\textbf{#1}}}
\newcommand{\BuiltInTok}[1]{#1}
\newcommand{\ExtensionTok}[1]{#1}
\newcommand{\PreprocessorTok}[1]{\textcolor[rgb]{0.56,0.35,0.01}{\textit{#1}}}
\newcommand{\AttributeTok}[1]{\textcolor[rgb]{0.77,0.63,0.00}{#1}}
\newcommand{\RegionMarkerTok}[1]{#1}
\newcommand{\InformationTok}[1]{\textcolor[rgb]{0.56,0.35,0.01}{\textbf{\textit{#1}}}}
\newcommand{\WarningTok}[1]{\textcolor[rgb]{0.56,0.35,0.01}{\textbf{\textit{#1}}}}
\newcommand{\AlertTok}[1]{\textcolor[rgb]{0.94,0.16,0.16}{#1}}
\newcommand{\ErrorTok}[1]{\textcolor[rgb]{0.64,0.00,0.00}{\textbf{#1}}}
\newcommand{\NormalTok}[1]{#1}
\usepackage{graphicx,grffile}
\makeatletter
\def\maxwidth{\ifdim\Gin@nat@width>\linewidth\linewidth\else\Gin@nat@width\fi}
\def\maxheight{\ifdim\Gin@nat@height>\textheight\textheight\else\Gin@nat@height\fi}
\makeatother
% Scale images if necessary, so that they will not overflow the page
% margins by default, and it is still possible to overwrite the defaults
% using explicit options in \includegraphics[width, height, ...]{}
\setkeys{Gin}{width=\maxwidth,height=\maxheight,keepaspectratio}
\IfFileExists{parskip.sty}{%
\usepackage{parskip}
}{% else
\setlength{\parindent}{0pt}
\setlength{\parskip}{6pt plus 2pt minus 1pt}
}
\setlength{\emergencystretch}{3em}  % prevent overfull lines
\providecommand{\tightlist}{%
  \setlength{\itemsep}{0pt}\setlength{\parskip}{0pt}}
\setcounter{secnumdepth}{5}
% Redefines (sub)paragraphs to behave more like sections
\ifx\paragraph\undefined\else
\let\oldparagraph\paragraph
\renewcommand{\paragraph}[1]{\oldparagraph{#1}\mbox{}}
\fi
\ifx\subparagraph\undefined\else
\let\oldsubparagraph\subparagraph
\renewcommand{\subparagraph}[1]{\oldsubparagraph{#1}\mbox{}}
\fi

%%% Use protect on footnotes to avoid problems with footnotes in titles
\let\rmarkdownfootnote\footnote%
\def\footnote{\protect\rmarkdownfootnote}

%%% Change title format to be more compact
\usepackage{titling}

% Create subtitle command for use in maketitle
\providecommand{\subtitle}[1]{
  \posttitle{
    \begin{center}\large#1\end{center}
    }
}

\setlength{\droptitle}{-2em}

  \title{Using the package skeleton for patient-level prediction studies}
    \pretitle{\vspace{\droptitle}\centering\huge}
  \posttitle{\par}
    \author{Jenna M. Reps}
    \preauthor{\centering\large\emph}
  \postauthor{\par}
      \predate{\centering\large\emph}
  \postdate{\par}
    \date{2019-11-08}


\begin{document}
\maketitle

{
\setcounter{tocdepth}{2}
\tableofcontents
}
\section{Introduction}\label{introduction}

This vignette describes how one can use the package skeleton for
patient-level prediction studies to create one's own study package. This
skeleton is aimed at patient-level prediction studies using the
\texttt{PatientLevelPrediction} package. The resulting package can be
used to execute the study at any site that has access to an
observational database in the Common Data Model. It will perform the
following steps:

\begin{enumerate}
\def\labelenumi{\arabic{enumi}.}
\tightlist
\item
  Instantiate all cohorts needed for the study in a study-specific
  cohort table.
\item
  The main analysis will be executed using the
  \texttt{PatientLevelPrediction} package, which involves development
  and internal validation of prediction models.
\item
  The prediction models can be exported into a network study package
  ready to share for external validation.
\end{enumerate}

The package skeleton currently implements an examplar study, predicting
various outcomes in multiple target populations. If desired (as a test),
one can run the package as is.

\subsection{Open the project in
Rstudio}\label{open-the-project-in-rstudio}

Make sure to have RStudio installed. Then open the R project downloaded
from ATLAS by decompressing the downloaded folder and clicking on the
.Rproj file (where is replaced by the study name you specified in
ATLAS). This should open an RStudio session.

\subsection{Installing all package
dependencies}\label{installing-all-package-dependencies}

Before you can build the package you downloaded from ATLAS you need to
make sure you have all the dependencies:

\begin{Shaded}
\begin{Highlighting}[]
\KeywordTok{source}\NormalTok{(}\StringTok{'./extras/packageDeps.R'}\NormalTok{)}
\end{Highlighting}
\end{Shaded}

\subsection{Building the package}\label{building-the-package}

Once you have the dependencies installed you can now build the R
package. This creates a library you can load to run the prediction
study. To build the package click `Build' on the top right hand side tab
menu (there are tabs: `Environment', `History', `Connections', `Build',
`Git'). Once in `Build' click the `Install and Restart' button. This
will now build your package and create the R library. If it succeeds you
will see `* DONE ()', if it fails you will see red output and the
library may not be created. Please report an issue to:
\url{https://github.com/OHDSI/PatientLevelPrediction/issues} if your
library does not get created.

\subsection{Running the package}\label{running-the-package}

To run the study, open the extras/CodeToRun.R R script (the file called
\texttt{CodeToRun.R} in the \texttt{extras} folder). This folder
specifies the R variables you need to define (e.g., outputFolder and
database connection settings). See the R help system for details:

\begin{Shaded}
\begin{Highlighting}[]
\KeywordTok{library}\NormalTok{(SkeletonpredictionStudy)}
\NormalTok{?execute}
\end{Highlighting}
\end{Shaded}

By default all the options are set to F for the execute fuction:

\begin{Shaded}
\begin{Highlighting}[]
\KeywordTok{execute}\NormalTok{(}\DataTypeTok{connectionDetails =}\NormalTok{ connectionDetails,}
        \DataTypeTok{cdmDatabaseSchema =} \StringTok{'your cdm schema'}\NormalTok{,}
        \DataTypeTok{cohortDatabaseSchema =} \StringTok{'your cohort schema'}\NormalTok{,}
        \DataTypeTok{cdmDatabaseName =} \StringTok{'Name of database used in study'}\NormalTok{,}
        \DataTypeTok{cohortTable =} \StringTok{"cohort"}\NormalTok{,}
        \DataTypeTok{oracleTempSchema =} \OtherTok{NULL}\NormalTok{,}
        \DataTypeTok{outputFolder =} \StringTok{'./results'}\NormalTok{,}
        \DataTypeTok{createProtocol =}\NormalTok{ F,}
        \DataTypeTok{createCohorts =}\NormalTok{ F,}
        \DataTypeTok{runAnalyses =}\NormalTok{ F,}
        \DataTypeTok{createResultsDoc =}\NormalTok{ F,}
        \DataTypeTok{packageResults =}\NormalTok{ F,}
        \DataTypeTok{createValidationPackage =}\NormalTok{ F,  }
        \CommentTok{#analysesToValidate = 1,}
        \DataTypeTok{minCellCount=} \DecValTok{5}\NormalTok{,}
        \DataTypeTok{createShiny =}\NormalTok{ F,}
        \DataTypeTok{createJournalDocument =}\NormalTok{ F,}
        \DataTypeTok{analysisIdDocument =} \DecValTok{1}\NormalTok{)}
\end{Highlighting}
\end{Shaded}

If you run the above nothing will happen as each option is false.

To create a study protocol set:

\begin{Shaded}
\begin{Highlighting}[]
\NormalTok{    createProtocol =}\StringTok{ }\NormalTok{T}
\end{Highlighting}
\end{Shaded}

This uses the settings you specified in ATLAS to generate a protocol for
the study.

To create the target and outcome cohorts (cohorts are created into
cohortDatabaseSchema.cohortTable)

\begin{Shaded}
\begin{Highlighting}[]
\NormalTok{    createCohorts =}\StringTok{ }\NormalTok{T}
\end{Highlighting}
\end{Shaded}

To develop and internally validate the models run the code:

\begin{Shaded}
\begin{Highlighting}[]
\NormalTok{    runAnalyses =}\StringTok{ }\NormalTok{T}
\end{Highlighting}
\end{Shaded}

If the study runs and you get results, you can then create a result
document with the full results appended to the protocol document by
running (this will fail if you have not run the study first):

\begin{Shaded}
\begin{Highlighting}[]
\NormalTok{    createResultsDoc =}\StringTok{ }\NormalTok{T}
\end{Highlighting}
\end{Shaded}

To package the results ready for sharing with others you can set:

\begin{Shaded}
\begin{Highlighting}[]
\NormalTok{    packageResults =}\StringTok{ }\NormalTok{T}
\end{Highlighting}
\end{Shaded}

To create a new R package that can be used to externally validate the
models you developed set (this will fail if you have not run the study
first to create models):

\begin{Shaded}
\begin{Highlighting}[]
\NormalTok{    createValidationPackage =}\StringTok{ }\NormalTok{T  }
    \CommentTok{#analysesToValidate = 1}
\end{Highlighting}
\end{Shaded}

If you do not set analysesToValidate then all the developed models will
be transproted into the validation R package. To restrict to Analysis 1
and 5 models set: \texttt{analysesToValidate\ =\ c(1,5)}. The validation
package will be found in your outputFolder directory with the same name
as your prediction package but with Validation appended (e.g.,
outputFolder/Validation). You can add this valdiation package directory
to the studyProtocol on the OHDSI github to share the model with other
collaborators.

Once you run the study you can view the results via a local shiny app by
running:

\begin{Shaded}
\begin{Highlighting}[]
    \KeywordTok{viewMultiplePlp}\NormalTok{(outputFolder) }
\end{Highlighting}
\end{Shaded}

however, if you want to create a shiny app that you can share with the
OHDSI community, you can run:

\begin{Shaded}
\begin{Highlighting}[]
\NormalTok{    createShiny =}\StringTok{ }\NormalTok{T  }
\end{Highlighting}
\end{Shaded}

This will create a directory in outputFolder named `ShinyApp'. You can
add this directory to the shinyDeploy OHDSI github to add it to the
website data.ohdsi.org . Any sensitive data are removed from the shiny
results and you can specify the minimum cell count for the shiny results
with `minCellCount', for example to only show results when there are 10
or more patients \texttt{minCellCount\ =\ 10}.

To create a template journal document for Analysis 3 run execute with:

\begin{Shaded}
\begin{Highlighting}[]
\NormalTok{    createJournalDocument =}\StringTok{ }\NormalTok{T}
\NormalTok{    analysisIdDocument =}\StringTok{ }\DecValTok{3}
\end{Highlighting}
\end{Shaded}

You ill then find the document in the outputFolder directory.

\subsection{extras/PackageMaintenance.R}\label{extraspackagemaintenance.r}

This file contains other useful code to be used only by the package
developer (you), such as code to generate the package manual, and code
to insert cohort definitions into the package. All statements in this
file assume the current working directory is set to the root of the
package.


\end{document}
